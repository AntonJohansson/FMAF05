\documentclass[a4paper]{article}

\usepackage[swedish]{babel}
\usepackage[utf8x]{inputenc}
\usepackage{amsmath}
\usepackage{amssymb}
\usepackage[T1]{fontenc}
\usepackage{graphicx}
\usepackage{epstopdf}
\usepackage{bm}
\usepackage{mathtools}
\usepackage{gauss}
\usepackage{placeins} %FloatBarrier

\newcommand{\mat}[1]{\bm{\mathit{#1}}}

\newcommand{\mline}{%
  \hspace{-\arraycolsep}%
  \strut\vrule
  \hspace{-\arraycolsep}%
}

\title{Komplettering}

\begin{document}

\maketitle

\section*{1.1}
\subsection*{c)}

\begin{equation*}
  \vec{x} = c_1e^{3t}\begin{bmatrix}6\\8\end{bmatrix} + c_2e^{2t}\begin{bmatrix}6\\9\end{bmatrix}
  \iff \begin{cases}
    x_1(t) = 6c_1e^{3t} + 6c_2e^{2t}\\
    x_2(t) = 8c_1e^{3t} + 9c_2e^{2t}
    \end{cases}
\end{equation*}
Om uttrycket för $x_1(t)$ och $x_2(t)$ båda förkortas med $e^{3t}$ fås
\begin{equation}
  \begin{cases}
    x_1(t) = 6c_1 + 6c_2e^{-t}\\
    x_2(t) = 8c_1 + 9c_2e^{-t}
\end{cases}\label{eq:11c_1}
\end{equation}
Då $t\to\infty$ går (\ref{eq:11c_1}) mot
\begin{equation*}
  \begin{cases}
    x_1(t) = 6c_1\\
    x_2(t) = 8c_1
\end{cases}
\end{equation*}
Kvoten $x_1(t)$/$x_2(t)$ blir därmed $\frac 34$ (för stora $t$). Denna kvot återfinns även som förhållandet mellan komponenterna i egenvektorn $s_1$.

\section*{1.3}
\subsection*{a)}

Enligt uppgiften gäller

\begin{equation*}
  \frac{\text{d}\vec{x}}{\text{d}t} = \mat{A}\vec{x} + \vec{f}(t), \quad \mat{A} = \begin{bmatrix}-11 & 6\\-12 & 6\end{bmatrix}, \vec{f}(t) = \begin{bmatrix}3\\4\end{bmatrix}e^{-it},
\end{equation*}

\noindent Notera att systemmatrisen $\mat{A}$ är den negativa systemmatrisen
från uppgift 1.1 a), egenvektorerna fås därmed som

\begin{equation*}
  \lambda = \begin{cases}-3\\-2\end{cases}
\end{equation*}

\noindent Eftersom egenvektorerna skiljer sig från insignalens frekvens så ger
sats 4.6 den stataionära lösningen till systemet

\begin{align*}
  \vec{x_P}(t) &= ((-i)\mat{I} - \mat{A})^{-1}\begin{bmatrix}3\\4\end{bmatrix}e^{-it} = \begin{bmatrix}11-i & -6\\12 & -6-i\end{bmatrix}^{-1}\begin{bmatrix}3\\4\end{bmatrix}e^{-it}\\[2ex]
  &= \frac{1}{5-5i}\begin{bmatrix}-6-i & 6\\ -12& 11 - i\end{bmatrix}\begin{bmatrix}3\\4\end{bmatrix}e^{-it}\\[2ex]
  &= \frac{1}{5-5i}\begin{bmatrix}6-3i\\8-4i\end{bmatrix}e^{-it}\\
  &= \begin{bmatrix}0.9+0.3i\\1.2 + 0.4i\end{bmatrix}
\end{align*}

\section*{1.4}

Det tidsdiskreta systemet för fågelpopulationen över tid ges som
\begin{equation*}
  \begin{cases}
    x_{n+1} = 0x_n + ky_n\\
    y_{n+1} = 0.8x_n + 0.6y_n
  \end{cases}, n = 0, 1, 2, \ldots
\end{equation*}
Differensekationerna för $x$ och $y$ beskriver i detta fall
derivatan av variablerna i diskreta tidsintervall på 1 år. Ur systemmatrisen
\begin{equation*}
  \mat{A} = \begin{bmatrix}0 & k \\ 0.8 & 0.6 \end{bmatrix}
\end{equation*}
fås egenvärdena som nollställena till det karekteristiska polynomet
(enligt sats 3.3)

\begin{align*}
  p_A(\lambda) &= (0-\lambda)(0.6 - \lambda) - 0.8k = 0 \iff \lambda^2 - 0.6\lambda - 0.8k = 0\\
               &\iff (\lambda - 0.3)^2 - 0.09 + 0.8k = 0\\
               &\iff \lambda = 0.3 \pm \sqrt{0.09 + 0.8k}.
\end{align*}

Enligt uppgiften är begynnelsetiståndet
\begin{equation*}
\begin{bmatrix}x_0\\y_0\end{bmatrix} = c_1\vec{s_1} + c_2\vec{s_2},\quad c_{1,2} \neq 0,
\end{equation*}
där $s_{1,2}$ är egenvektorerna till systemmatrisen $\mat{A}$.

Eftersom varje år motsvaras av att multiplicera det föregående årets populationstillstånd med systemmatrisen $\mat{A}$, ges fågelpopulationen år $n$ av
\begin{equation*}
\begin{bmatrix}x_n\\y_n\end{bmatrix} = \mat{A}^n\begin{bmatrix}x_0\\y_0\end{bmatrix}, = c_1\mat{A}^n\vec{s_1} + c_2\mat{A}^n\vec{s_2}
\end{equation*}
men eftersom $s_{1,2}$ är egenvektorer till $\mat{A}$ kan transformationen $\mat{A}^ns_{1,2}$ även skrivas som $\lambda_{1,2}^ns_{1,2}$ enligt defintionen av egenvektorer och egenvärden, d.v.s
\begin{equation}
\begin{bmatrix}x_n\\y_n\end{bmatrix} =  c_1\lambda_1^n\vec{s_1} + c_2\lambda_2\vec{s_2}\label{eq:14_key}
\end{equation}

\subsection*{a)}

$k = \frac 15$ ger egenvärdena $\lambda_1 = 0.8$ och $\lambda_2 = -0.2$. Eftersom $|\lambda_{1,2}| < 1$ ger ekvation (\ref{eq:14_key}) att fågelpopulationens storlek går mot noll oavsett värdena på egenvektorerna av $\mat{A}$.

\subsection*{b)}

På samma sätt som i förra uppgiften ger $k = \frac 12$ egenvärdena $\lambda_1 =
1$ och $\lambda_2 = -0.4$. Eftersom $|\lambda_2| < 1$ kommer termen $c_2\lambda_2^n\vec{s_2} \to 0, n \to \infty$ (i ekv. (\ref{eq:14_key})). Därmed behöver enbart egenvektorn för $\lambda_1$ bestämmas
\begin{align*}
  (\mat{A} - \lambda_1\mat{I})\vec{s_1} = 0 &\Rightarrow
  \begin{gmatrix}[p]
    -1 & 0.5 & \mline & 0\\
    0.8 & -0.4 & \mline & 0
    \rowops
    \add[0.8]{0}{1}
  \end{gmatrix}
  \Rightarrow
  \begin{gmatrix}[p]
    -1 & 0.5 & \mline & 0\\
    0 & 1 & \mline & t
    \rowops
    \add[-0.5]{1}{0}
  \end{gmatrix}\\
  &\Rightarrow
  \begin{gmatrix}[p]
    -1 & 0 & \mline & -0.5t\\
    0 & 1 & \mline & t
  \end{gmatrix}\\
  &\Rightarrow \vec{s_1} = t_1\begin{bmatrix}1\\2\end{bmatrix}
\end{align*}
Slutligen ger ekvation (\ref{eq:14_key}) att
\begin{equation*}
\begin{bmatrix}x_n\\y_n\end{bmatrix} =c_1 1^n\begin{bmatrix}1\\2\end{bmatrix} + c_2\lambda_2\vec{s_2}
\end{equation*}
Populationen kommer efter en tid att stabiliseras vid $x = c_1$, $y = 2c_1$, d.v.s. det finns dubbelt så många vuxna som kycklingar.

\subsection*{c)}

Här ger $k = 2$ egenvärdena $\lambda_1 = 1.6$ och $\lambda_2 = -1$. Först bestäms egenvektorn för $\lambda_1$
\begin{align*}
  (\mat{A} - \lambda_1\mat{I})\vec{s_1} = 0 &\Rightarrow
  \begin{gmatrix}[p]
    -1.6 & 2 & \mline & 0\\
    0.8 & -1 & \mline & 0
    \rowops
    \add[0.5]{0}{1}
  \end{gmatrix}
  \Rightarrow
  \begin{gmatrix}[p]
    -1.6 & 2 & \mline & 0\\
    0 & 1 & \mline & t
    \rowops
    \add[-2]{1}{0}
  \end{gmatrix}\\
  &\Rightarrow
  \begin{gmatrix}[p]
    -1.6 & 0 & \mline & -2t\\
    0 & 1 & \mline & t
  \end{gmatrix}\\
  &\Rightarrow \vec{s_1} = t\begin{bmatrix}5\\4\end{bmatrix}.
\end{align*}
Därefter bestäms egenvektorn för $\lambda_2$
\begin{align*}
  (\mat{A} - \lambda_2\mat{I})\vec{s_2} = 0 &\Rightarrow
  \begin{gmatrix}[p]
    1 & 2 & \mline & 0\\
    0.8 & 1.6 & \mline & 0
    \rowops
    \add[-0.8]{0}{1}
  \end{gmatrix}
  \Rightarrow
  \begin{gmatrix}[p]
    1 & 2 & \mline & 0\\
    0 & 1 & \mline & t
    \rowops
    \add[-2]{1}{0}
  \end{gmatrix}\\
  &\Rightarrow
  \begin{gmatrix}[p]
    1 & 0 & \mline & -2t\\
    0 & 1 & \mline & t
  \end{gmatrix}\\
  &\Rightarrow \vec{s_2} = t\begin{bmatrix}-2\\1\end{bmatrix}.
\end{align*}
Ekvation (\ref{eq:14_key}) ger tillsist lösningen
\begin{equation*}
\begin{bmatrix}x_n\\y_n\end{bmatrix} =c_1 1.6^n\begin{bmatrix}5\\4\end{bmatrix} + c_2(-1)^n\begin{bmatrix}-2\\1\end{bmatrix}
\end{equation*}
Den andra termen kommer variera mellan konstanta värden, medan den första termen växer obegränsad då $n\to\infty$. Alltså kommer populationen att växa obegränsat.

\end{document}
